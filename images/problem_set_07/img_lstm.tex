\documentclass[class=minimal,border=10pt]{standalone}

% пакеты для математики
\usepackage{amsmath,amsfonts,amssymb,amsthm,mathtools}  
\mathtoolsset{showonlyrefs=true}  % Показывать номера только у тех формул, на которые есть \eqref{} в тексте.

\usepackage[british,russian]{babel} % выбор языка для документа
\usepackage[utf8]{inputenc}          % utf8 кодировка

% Основные шрифты 
% \usepackage{fontspec}         
% \setmainfont{Linux Libertine O}  % задаёт основной шрифт документа

% Математические шрифты 
\usepackage{unicode-math}     
\setmathfont[math-style=upright]{[Neo Euler.otf]} 

\usepackage{tikz}
\usepackage{xcolor}

% мои цвета https://www.artlebedev.ru/colors/
\definecolor{titleblue}{rgb}{0.2,0.4,0.6} 
% \definecolor{blue}{rgb}{0.2,0.4,0.6} 
% \definecolor{red}{rgb}{1,0,0.2} 
% \definecolor{green}{rgb}{0,0.6,0} 
\definecolor{purp}{rgb}{0.4,0,0.8} 

\definecolor{blue}{RGB}{0,114,178}
\definecolor{red}{RGB}{213,94,0}
\definecolor{yellow}{RGB}{240,228,66}
\definecolor{green}{RGB}{0,128, 0}

\definecolor{amethyst}{rgb}{0.6, 0.4, 0.8}
\definecolor{junglegreen}{rgb}{0.16, 0.67, 0.53}


% цвета из geogebra 
\definecolor{litebrown}{rgb}{0.6,0.2,0}
\definecolor{darkbrown}{rgb}{0.75,0.75,0.75}

\usetikzlibrary{calc,shapes,positioning}

\begin{document}
	% \definecolor{zzttqq}{rgb}{0.6,0.2,0.}
	% \begin{tikzpicture}[scale=0.6]
	% 		\fill[line width=2.pt,color=zzttqq,fill=zzttqq,fill opacity=0.10000000149011612] (2.,8.) -- (2.,6.) -- (4.,6.) -- (4.,8.) -- cycle;
	% 		\draw  (3.,3.) circle (1.cm);
	% 		\draw [color=zzttqq] (2.,8.)-- (2.,6.);
	% 		\draw [color=zzttqq] (2.,6.)-- (4.,6.);
	% 		\draw [color=zzttqq] (4.,6.)-- (4.,8.);
	% 		\draw [color=zzttqq] (4.,8.)-- (2.,8.);
	% 		\draw (3,11) circle (1.cm);
	% 		\draw [->] (3,4) -- (3,6); 
	% 		\draw [->] (3,8) -- (3,10);
	% 		\draw [shift={(4.45,7.)}]  plot[domain=-2.541806576912956:2.541806576912956,variable=\t]({1.*1.7715861960215864*cos(\t r)+0.*1.7715861960215864*sin(\t r)},{0.*1.7715861960215864*cos(\t r)+1.*1.7715861960215864*sin(\t r)});
	% 		\draw [->] (3.4,5.6) -- (3,6.);
	% 		\draw (2.2,3.5) node[anchor=north west] {$y_{t-1}$};
	% 		\draw (2.4,11.5) node[anchor=north west] {$y_t$};
	% 		\draw (6.3,7.8) node[anchor=north east] {$0.5$};
	% 		\draw (2.4,7.6) node[anchor=north west] {$h_t$};
	% 		\draw (1.7,9.7) node[anchor=north west] {$0.5$};
	% 		\draw (1.7,5.6) node[anchor=north west] {$0.5$};
	% \end{tikzpicture}



% 		\begin{center}
% 		\begin{tikzpicture}
% 			\tikzstyle{place}=[circle, draw=black, minimum size = 12mm]
% 			\tikzstyle{placeh}=[minimum height=32pt,minimum width=32pt, inner sep=2pt, draw=black]
			
% 			\node at (-1.2,0) (here) {};
			
% 			\draw node at (0, 2) [place, fill=blue, opacity=0.1] (y) {$\hat y_{t-2}$};
% 			\draw node at (0, 2) [place] {$\hat y_{t-2}$};  % мои костыли максимально всратые
% 			\draw node at (0, 0) [placeh, fill=red, opacity=0.1] (h) {$h_{t-2}$};
% 			\draw node at (0, 0) [placeh] {$h_{t-2}$};
% 			\draw node at (0, -2) [place, fill=green, opacity=0.1] (x) {$x_{t-2}$};
% 			\draw node at (0, -2) [place] {$x_{t-2}$};
				
% 			\draw [->]  (here) to node[above]{$W$} (h);
% 			\draw [->]  (x) to node[left]{$V$} (h);
% 			\draw [->]  (h) to node[left]{$U$} (y);
			
% 			\draw node at (2, 2) [place, fill=blue, opacity=0.1] (y1) {$\hat y_{t-1}$};
% 			\draw node at (2, 2) [place] {$\hat y_{t-1}$};  % мои костыли максимально всратые
% 			\draw node at (2, 0) [placeh, fill=red, opacity=0.1] (h1) {$h_{t-1}$};
% 			\draw node at (2, 0) [placeh] {$h_{t-1}$};
% 			\draw node at (2, -2) [place, fill=green, opacity=0.1] (x1) {$x_{t-1}$};
% 			\draw node at (2, -2) [place] {$x_{t-1}$};
			
% 			\draw [->]  (h) to node[above]{$W$} (h1);
% 			\draw [->]  (x1) to node[left]{$V$} (h1);
% 			\draw [->]  (h1) to node[left]{$U$} (y1);
			
% 			\draw node at (4, 2) [place, fill=blue, opacity=0.1] (y2) {$\hat y_{t}$};
% 			\draw node at (4, 2) [place] {$\hat y_{t}$};  % мои костыли максимально всратые
% 			\draw node at (4, 0) [placeh, fill=red, opacity=0.1] (h2) {$h_{t}$};
% 			\draw node at (4, 0) [placeh] {$h_{t}$};
% 			\draw node at (4, -2) [place, fill=green, opacity=0.1] (x2) {$x_{t}$};
% 			\draw node at (4, -2) [place] {$x_{t}$};
			
% 			\draw [->]  (h1) to node[above]{$W$} (h2);
% 			\draw [->]  (x2) to node[left]{$V$} (h2);
% 			\draw [->]  (h2) to node[left]{$U$} (y2);
			
			
% 			\draw node at (6, 2) [place, fill=blue, opacity=0.1] (y3) {$\hat y_{t+1}$};
% 			\draw node at (6, 2) [place] {$\hat y_{t+1}$};  % мои костыли максимально всратые
% 			\draw node at (6, 0) [placeh, fill=red, opacity=0.1] (h3) {$h_{t+1}$};
% 			\draw node at (6, 0) [placeh] {$h_{t+1}$};
% 			\draw node at (6, -2) [place, fill=green, opacity=0.1] (x3) {$x_{t+1}$};
% 			\draw node at (6, -2) [place] {$x_{t+1}$};
			
% 			\draw [->]  (h2) to node[above]{$W$} (h3);
% 			\draw [->]  (x3) to node[left]{$V$} (h3);
% 			\draw [->]  (h3) to node[left]{$U$} (y3);
			
% 			\node at (7.2,0) (end) {};
% 			\draw [->]  (h3) to node[above]{$W$} (end);
% 		\end{tikzpicture}
% 	\end{center}
	
	
% https://tex.stackexchange.com/questions/432312/how-do-i-draw-an-lstm-cell-in-tikz/432344



	
\end{document}