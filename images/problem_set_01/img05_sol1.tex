\documentclass[class=minimal,border=10pt]{standalone}

% пакеты для математики
\usepackage{amsmath,amsfonts,amssymb,amsthm,mathtools}  
\mathtoolsset{showonlyrefs=true}  % Показывать номера только у тех формул, на которые есть \eqref{} в тексте.

\usepackage[british,russian]{babel} % выбор языка для документа
\usepackage[utf8]{inputenc}          % utf8 кодировка

% Основные шрифты 
% \usepackage{fontspec}         
% \setmainfont{Linux Libertine O}  % задаёт основной шрифт документа

% Математические шрифты 
\usepackage{unicode-math}     
\setmathfont[math-style=upright]{[Neo Euler.otf]} 

\usepackage{tikz}
\usepackage{xcolor}

% мои цвета https://www.artlebedev.ru/colors/
\definecolor{titleblue}{rgb}{0.2,0.4,0.6} 
% \definecolor{blue}{rgb}{0.2,0.4,0.6} 
% \definecolor{red}{rgb}{1,0,0.2} 
% \definecolor{green}{rgb}{0,0.6,0} 
\definecolor{purp}{rgb}{0.4,0,0.8} 

\definecolor{blue}{RGB}{0,114,178}
\definecolor{red}{RGB}{213,94,0}
\definecolor{yellow}{RGB}{240,228,66}
\definecolor{green}{RGB}{0,128, 0}

\definecolor{amethyst}{rgb}{0.6, 0.4, 0.8}
\definecolor{junglegreen}{rgb}{0.16, 0.67, 0.53}


% цвета из geogebra 
\definecolor{litebrown}{rgb}{0.6,0.2,0}
\definecolor{darkbrown}{rgb}{0.75,0.75,0.75}

\usetikzlibrary{calc,shapes,positioning}

% рисование крестов
% https://tex.stackexchange.com/questions/123760/draw-crosses-in-tikz
\tikzset{
    cross/.pic = {
    \draw[line width=2.pt, rotate = 45] (-#1,0) -- (#1,0);
    \draw[line width=2.pt, rotate = 45] (0,-#1) -- (0, #1);
    }
}

\begin{document}
    \begin{minipage}{0.29\linewidth}
    \begin{center}
        \begin{tikzpicture}[line cap=round,line join=round,x=1.0cm,y=1.0cm]
        
        \draw [->, line width=1.pt] (-0.3,0) --(3.5,0); 
        \draw [->, line width=1.pt] (0,-0.3) --(0,3.5); 
        
        \draw [fill=blue] (0,0) circle (4pt);
        \draw [fill=blue] (0,2) circle (4pt);
        \draw [fill=blue] (2,0) circle (4pt);
        \path (2,2) pic[red] {cross=4pt};
        \draw [line width=2.pt,dash pattern=on 3pt off 3pt] (-0.4,3.2)-- (3.2,-0.4);
        \end{tikzpicture}
    \end{center}
    \end{minipage}
    \hfill
    \begin{minipage}{0.59\linewidth}
    \begin{center}
    \begin{tikzpicture}[line cap=round,line join=round,x=1.0cm,y=1.0cm]
    \clip(-4,0) rectangle (3,4.5);
    \draw [line width=1.pt] (-3,4) circle (0.5cm) node {$1$};
    \draw [line width=1.pt] (-3,2.5) circle (0.5cm) node {$x_1$};
    \draw [line width=1.pt] (-3,1) circle (0.5cm) node {$x_2$};

    \draw [line width=1.pt] (-1,3)--(-1,2)--(1.5,2)--(1.5,3)--cycle;
    \draw [line width=1.pt] (-0.6,2.2)--(0.2,2.2)--(0.2,2.8)--(1,2.8);
    \draw (0.3,1.7) node {$\gamma = 0$};

    \draw [->, line width=1.pt] (-2.5,4)--(-1.1,2.7) node[pos=0.5,above] {\small $-\tfrac{3}{2}$};
    \draw [->, line width=1.pt] (-2.5,2.5)--(-1.1,2.5) node[pos=0.3,above] {\small $1$};
    \draw [->, line width=1.pt] (-2.5,1)--(-1.1,2.3) node[pos=0.5,below] {\small $1$};
    \draw [->, line width=1.pt] (1.5,2.5)--(2.5,2.5) node[pos=1,right] {$\hat y$};
    \end{tikzpicture}
    \end{center}
    \end{minipage}
\end{document}
