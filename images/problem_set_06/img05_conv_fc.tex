\documentclass[class=minimal,border=10pt]{standalone}

% пакеты для математики
\usepackage{amsmath,amsfonts,amssymb,amsthm,mathtools}  
\mathtoolsset{showonlyrefs=true}  % Показывать номера только у тех формул, на которые есть \eqref{} в тексте.

\usepackage[british,russian]{babel} % выбор языка для документа
\usepackage[utf8]{inputenc}          % utf8 кодировка

% Основные шрифты 
% \usepackage{fontspec}         
% \setmainfont{Linux Libertine O}  % задаёт основной шрифт документа

% Математические шрифты 
\usepackage{unicode-math}     
\setmathfont[math-style=upright]{[Neo Euler.otf]} 

\usepackage{tikz}
\usepackage{xcolor}

% мои цвета https://www.artlebedev.ru/colors/
\definecolor{titleblue}{rgb}{0.2,0.4,0.6} 
% \definecolor{blue}{rgb}{0.2,0.4,0.6} 
% \definecolor{red}{rgb}{1,0,0.2} 
% \definecolor{green}{rgb}{0,0.6,0} 
\definecolor{purp}{rgb}{0.4,0,0.8} 

\definecolor{blue}{RGB}{0,114,178}
\definecolor{red}{RGB}{213,94,0}
\definecolor{yellow}{RGB}{240,228,66}
\definecolor{green}{RGB}{0,128, 0}

\definecolor{amethyst}{rgb}{0.6, 0.4, 0.8}
\definecolor{junglegreen}{rgb}{0.16, 0.67, 0.53}


% цвета из geogebra 
\definecolor{litebrown}{rgb}{0.6,0.2,0}
\definecolor{darkbrown}{rgb}{0.75,0.75,0.75}

\usetikzlibrary{calc,shapes,positioning}

\begin{document}
    \begin{tikzpicture}[scale=1]
        \tikzstyle{place}=[circle, draw=black, minimum size = 12mm]
        \tikzstyle{placeh}=[draw=black, minimum height=25pt,minimum width=60pt,inner sep=2pt]
        
        % Input
        \draw node at (0, -1*1.5) [place] (first_1) {$x_{11}$};
        \draw node at (0, -2*1.5) [place] (first_2) {$x_{12}$};
        \draw node at (0, -3*1.5) [place] (first_3) {$x_{13}$};
        \draw node at (0, -4*1.5) [place] (first_4) {$x_{14}$};
        \draw node at (0, -5*1.5) [place] (first_5) {$x_{21}$};
        
        \draw node at (-0.2, -8.5) {{\LARGE $\ldots$}};
        
        % Hidden 1
        \node at (5, -1*1.8) [placeh] (second_1){$a$};
        \node at (5, -2*1.8) [placeh] (second_2){$b$};
        \node at (5, -3*1.8) [placeh] (second_3){$c$};
        \node at (5, -4*1.8) [placeh] (second_4){$d$};
        
        \draw [->]  (first_1) to node[above]{} (second_1);
        \draw [->]  (first_1) to node[above]{} (second_2);
        \draw [->]  (first_1) to node[above]{} (second_3);
        \draw [->]  (first_1) to node[above]{} (second_4);
        
        \draw [->]  (first_2) to node[below]{} (second_1);
        \draw [->]  (first_2) to node[below]{} (second_2);
        \draw [->]  (first_2) to node[below]{} (second_3);
        \draw [->]  (first_2) to node[below]{} (second_4);

        \draw [->]  (first_3) to node[below]{} (second_1);
        \draw [->]  (first_3) to node[below]{} (second_2);
        \draw [->]  (first_3) to node[below]{} (second_3);
        \draw [->]  (first_3) to node[below]{} (second_4);
        
        \draw [->]  (first_4) to node[below]{} (second_1);
        \draw [->]  (first_4) to node[below]{} (second_2);
        \draw [->]  (first_4) to node[below]{} (second_3);
        \draw [->]  (first_4) to node[below]{} (second_4);
        
        \draw [->]  (first_5) to node[below]{} (second_1);
        \draw [->]  (first_5) to node[below]{} (second_2);
        \draw [->]  (first_5) to node[below]{} (second_3);
        \draw [->]  (first_5) to node[below]{} (second_4);
    \end{tikzpicture}
\end{document}
