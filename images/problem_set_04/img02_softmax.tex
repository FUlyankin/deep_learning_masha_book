\documentclass[class=minimal,border=10pt]{standalone}

% пакеты для математики
\usepackage{amsmath,amsfonts,amssymb,amsthm,mathtools}  
\mathtoolsset{showonlyrefs=true}  % Показывать номера только у тех формул, на которые есть \eqref{} в тексте.

\usepackage[british,russian]{babel} % выбор языка для документа
\usepackage[utf8]{inputenc}          % utf8 кодировка

% Основные шрифты 
% \usepackage{fontspec}         
% \setmainfont{Linux Libertine O}  % задаёт основной шрифт документа

% Математические шрифты 
\usepackage{unicode-math}     
\setmathfont[math-style=upright]{[Neo Euler.otf]} 

\usepackage{tikz}
\usepackage{xcolor}

% мои цвета https://www.artlebedev.ru/colors/
\definecolor{titleblue}{rgb}{0.2,0.4,0.6} 
% \definecolor{blue}{rgb}{0.2,0.4,0.6} 
% \definecolor{red}{rgb}{1,0,0.2} 
% \definecolor{green}{rgb}{0,0.6,0} 
\definecolor{purp}{rgb}{0.4,0,0.8} 

\definecolor{blue}{RGB}{0,114,178}
\definecolor{red}{RGB}{213,94,0}
\definecolor{yellow}{RGB}{240,228,66}
\definecolor{green}{RGB}{0,128, 0}

\definecolor{amethyst}{rgb}{0.6, 0.4, 0.8}
\definecolor{junglegreen}{rgb}{0.16, 0.67, 0.53}


% цвета из geogebra 
\definecolor{litebrown}{rgb}{0.6,0.2,0}
\definecolor{darkbrown}{rgb}{0.75,0.75,0.75}

\usetikzlibrary{calc,shapes,positioning}

\begin{document}
\begin{tikzpicture}[scale=1]
    \tikzstyle{place}=[circle, draw=black, minimum size = 12mm]
    \tikzstyle{placeh}=[draw=black, minimum height=25pt,minimum width=60pt,inner sep=2pt]
    
    % Input
    \draw node at (0, -1*1.5) [place] (first_1) {$x_1$};
    \draw node at (0, -2*1.5) [place] (first_2) {$x_2$};
    \draw node at (0, -3*1.5) [place] (first_3) {$x_3$};
    \draw node at (0, -4.2*1.5) [place] (first_4) {$x_k$};
    \draw node at (0, -5.4) {{\LARGE $\ldots$}};
    
    % Hidden 1
    \node at (5, -1*1.8) [placeh] (second_1){$z_1$};
    \node at (5, -2*1.8) [placeh] (second_2){$z_2$};
    \node at (5, -3*1.8) [placeh] (second_3){$z_3$};
    
    \node at (9, -1*1.8) [placeh] (th_1){$\frac{e^{z_1}}{e^{z_1} + e^{z_2} + e^{z_3}}$};
    \node at (9, -2*1.8) [placeh] (th_2){$\frac{e^{z_2}}{e^{z_1} + e^{z_2} + e^{z_3}}$};
    \node at (9, -3*1.8) [placeh] (th_3){$\frac{e^{z_3}}{e^{z_1} + e^{z_2} + e^{z_3}}$};
    
    \draw [->]  (first_1) to node[above]{} (second_1);
    \draw [->]  (first_1) to node[above]{} (second_2);
    \draw [->]  (first_1) to node[above]{} (second_3);

    \draw [->]  (first_2) to node[below]{} (second_1);
    \draw [->]  (first_2) to node[below]{} (second_2);
    \draw [->]  (first_2) to node[below]{} (second_3);

    \draw [->]  (first_3) to node[below]{} (second_1);
    \draw [->]  (first_3) to node[below]{} (second_2);
    \draw [->]  (first_3) to node[below]{} (second_3);

    \draw [->]  (first_4) to node[below]{} (second_1);
    \draw [->]  (first_4) to node[below]{} (second_2);
    \draw [->]  (first_4) to node[below]{} (second_3);
    
    \draw [->]  (second_1) to node[below]{} (th_1);
    \draw [->]  (second_1) to node[below]{} (th_2);
    \draw [->]  (second_1) to node[below]{} (th_3);
    \draw [->]  (second_2) to node[below]{} (th_1);
    \draw [->]  (second_2) to node[below]{} (th_2);
    \draw [->]  (second_2) to node[below]{} (th_3);
    \draw [->]  (second_3) to node[below]{} (th_1);
    \draw [->]  (second_3) to node[below]{} (th_2);
    \draw [->]  (second_3) to node[below]{} (th_3);
    
    \fill[line width=1.pt ,color=green, fill=blue ,fill opacity=0.8] (11,-1.4) -- (11,-2.2) -- (13,-2.2) -- (13,-1.4) -- cycle;
    \fill[line width=1.pt ,color=green, fill=blue ,fill opacity=0.8] (11,-4) -- (11,-3.2) -- (14,-3.2) -- (14,-4) -- cycle;
    \fill[line width=1.pt ,color=green, fill=blue ,fill opacity=0.8] (11,-5.8) -- (11,-5) -- (12,-5) -- (12,-5.8) -- cycle;
    
    \draw [line width=1.pt, ->]  (11,-0.7) -- (11, -6.5);
    \draw [line width=1.pt, ->]  (10.7,-1) -- (14, -1);
\end{tikzpicture}
\end{document}
