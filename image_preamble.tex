\documentclass[class=minimal,border=10pt]{standalone}

% пакеты для математики
\usepackage{amsmath,amsfonts,amssymb,amsthm,mathtools}  
\mathtoolsset{showonlyrefs=true}  % Показывать номера только у тех формул, на которые есть \eqref{} в тексте.

\usepackage[british,russian]{babel} % выбор языка для документа
\usepackage[utf8]{inputenc}          % utf8 кодировка

% Основные шрифты 
% \usepackage{fontspec}         
% \setmainfont{Linux Libertine O}  % задаёт основной шрифт документа

% Математические шрифты 
\usepackage{unicode-math}     
\setmathfont[math-style=upright]{[Neo Euler.otf]} 

\usepackage{tikz}
\usepackage{xcolor}

% мои цвета https://www.artlebedev.ru/colors/
\definecolor{titleblue}{rgb}{0.2,0.4,0.6} 
% \definecolor{blue}{rgb}{0.2,0.4,0.6} 
% \definecolor{red}{rgb}{1,0,0.2} 
% \definecolor{green}{rgb}{0,0.6,0} 
\definecolor{purp}{rgb}{0.4,0,0.8} 

\definecolor{blue}{RGB}{0,114,178}
\definecolor{red}{RGB}{213,94,0}
\definecolor{yellow}{RGB}{240,228,66}
\definecolor{green}{RGB}{0,128, 0}

\definecolor{amethyst}{rgb}{0.6, 0.4, 0.8}
\definecolor{junglegreen}{rgb}{0.16, 0.67, 0.53}


% цвета из geogebra 
\definecolor{litebrown}{rgb}{0.6,0.2,0}
\definecolor{darkbrown}{rgb}{0.75,0.75,0.75}

\usetikzlibrary{calc,shapes,positioning}

% рисование крестов
% https://tex.stackexchange.com/questions/123760/draw-crosses-in-tikz
\tikzset{
    cross/.pic = {
    \draw[line width=2.pt, rotate = 45] (-#1,0) -- (#1,0);
    \draw[line width=2.pt, rotate = 45] (0,-#1) -- (0, #1);
    }
}
